%\documentclass[notes=show]{beamer} 
\documentclass[]{beamer}
\usetheme{umbc1}  
\useoutertheme{umbcfootline}
\setbeamertemplate{navigation symbols}{}  % see Remark below 
\setfootline{\insertshortauthor \hfill \insertsubtitle
    \hfill slide \insertframenumber/\inserttotalframenumber} 


\title{Bachelorarbeit}
\subtitle{Indentifikation von fehlerbehafteten Daten zur energiebewu\ss ten Programmierung}
\author[R. Frank]{Ren\'{e} Frank}
\institute[Ren\'{e} Frank]{
  Philipps-Universit\"at Marburg\\ 
  FB Mathematik \& Informatik\\[1ex]
  \texttt{frank@mathematik.uni-marburg.de}
}
\date{\today}

\usepackage[T1]{fontenc}
\usepackage[utf8]{inputenc}
\usepackage[ngerman]{babel}

%% use of boxes
\usepackage{fancybox}
\useinnertheme{umbcboxes}
\setbeamercolor{umbcboxes}{bg=violet!15,fg=black}

%% paste grafics
\usepackage{graphics}

%% tikz
\usepackage{tikz}
\usetikzlibrary{arrows,shadows,patterns} % for pgf-umlsd

%% framed
\usepackage{framed}


%% color package
\usepackage{color}

%% dashrule
\usepackage{dashrule}

%% different font family
\usepackage{helvet} %

%% listing
\usepackage{listings}

%% uml
\usepackage{uml}


%% set the items after \pause of one slide transparent 
\setbeamercovered{transparent}

\newcommand{\dashrule}[0]{\hdashrule{\linewidth}{1pt}{3pt}}

\newcommand{\partPage}[1]{
	{
	\setbeamertemplate{background canvas}{%
	\begin{tikzpicture}
    		\clip (0,0) rectangle (\paperwidth,\paperheight);
    		\fill[color=tmpBlue] (\paperwidth-10pt,0) rectangle (\paperwidth,\paperheight);
	\end{tikzpicture}}
	#1
	}
}

\defbeamertemplate*{part page}{borstel}[1][]{ 
 
  \begin{centering}
    \vskip1em\par
    \begin{beamercolorbox}[sep=8pt,center,#1]{}
      	\begin{myblock}
      		\usebeamerfont{part title}\usebeamercolor*[fg]{part name}\insertpart\par
      	\end{myblock}
    \end{beamercolorbox}
  \end{centering}
} 
\setbeamertemplate{part page}[borstel][]

\definecolor{llightGray}{HTML}{C0C0C0}
\definecolor{Sand}{HTML}{FFFFCC}
\definecolor{LLightBlue}{HTML}{CCCCFF}
\definecolor{LLila}{HTML}{870055}
\definecolor{DarkGrey}{rgb}{0.1,0.1,0.1}
\definecolor{tmpBlue}{HTML}{9999D9}

\newtheorem{mydef}{Strategien}

\newsavebox\blockbox
\newenvironment{myblock}{%
  \begin{lrbox}{\blockbox}%
    \begin{minipage}{.8\textwidth}
}{
    \end{minipage}
  \end{lrbox}
  \tikz\node[
    draw=black,
    line width=0.5pt,
    inner sep=10pt,
    outer sep=0pt,
  ]{\usebox\blockbox};
}

\lstdefinestyle{stJava}{basicstyle=\scriptsize\ttfamily,
	classoffset=0,	
	numbers=left, 				% show line numbers on the left
	stepnumber=1, 				% show every  line number
	numberstyle=\scriptsize, 			% font sitze of line numbers
	numbersep=5pt, 				% separation to the text
	language=Java,				% language of source
	backgroundcolor={\color{Sand}},
	stringstyle=\color{blue},		% style of strings
	emphstyle=\color{red},			% sytle of ???
	commentstyle=\color{gray},		% style of comments
	keywordstyle=\bfseries,			% style of keywords
	captionpos=b,
	tabsize=4,				% indentation
	showspaces=false,			% show spaces
	showtabs=false,				% show tabs
	showstringspaces=false,			% show spaces in strings
	breaklines=true,			% breaks long lines
	keywordstyle=\color{LLila},          % keyword style
	morekeywords={enum},
  	%stringstyle=\color{mauve},         % string literal style
  	escapeinside={\%*}{*)},           % if you want to add LaTeX within your code
  	classoffset=1,
	morekeywords={RANDOM, LOSS, ZERO, BITFLIP, BITFLIPB, NONE},keywordstyle=\color{blue},
	classoffset=0
	%frame=tbl 				% can be: none, single, trbl (or trBL, ...), shadowbox	      		
%    keywordstyle=\color{blue}\bfseries,
}