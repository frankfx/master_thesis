\subsection{Dynamische Konfiguration}

Die Klasse \courier{AddRunTimeAnnotation} ist f\"ur die dynamische Anpassung der Fehlerwerte zur Laufzeit verantwortlich. Annotationen sind urspr\"unglich nur Metainformationen, die im Quelltext eines Programmes notiert werden und zusätzlich semantische Informationen bereit stellen. Problematisch an der Verwendung von Annotationen ist die Tatsache, dass sich diese nicht ohne weiteres dynamisch verändern lassen. Aus diesem Grund wurden verschiedene L\"osungswege getestet und schlie\ss lich eine Manipulation auf Bytecode-Ebene durchgef\"uhrt. Zu diesem Zweck wurde sich die Funktionalit\"at der Bibliotheken Javassist und Tools zunutze gemacht, welche im Kapitel Grundlagen bzw. deren genaue Konfiguration im Abschnitt Voraussetzungen/Libraries n\"aher erl\"autert werden.\\
Bei der Umsetzung der dynamischen Regulierung der Fehlerwerte wurde in der Klasse\\ \courier{AddRunTimeAnnotation} eine statische Methode zur Modifikation des  StreamProcessor's geschrieben. Aufgerufen wird die Methode direkt über den \courier{Controller}. Sie verändert bei ihrer Ausführung den Class-File des StreamProcessor's, lädt ihn in die JVM und liefert im Anschluss eine Instanz dieser Klasse an den \courier{Controller} zur\"uck. Die Beziehungen der einzelnen Klassen sind in Abbildung \ref{AddRunTimeAnnotationUML} grafisch dargestellt. 

\begin{figure}[!htb]
\centering
		\umlDiagram[box=,border,sizeX=12cm,sizeY=7cm,ref=pack]{		
			\umlClass[pos=\umlTop{pack}, stereotype=Class, posDelta={0, -1},
				refpoint=t]{AddRunTimeAnnotation}{}{}
			\umlClass[pos=\umlBottomLeft{pack}, stereotype=Class, posDelta={5, 5},
				refpoint=t]{FileProcessor}{}{}
			\umlClass[pos=\umlTopRight{pack}, stereotype=Annotation, posDelta={-5, -5},
				refpoint=t]{FaultInj}{}{}	
			\umlClass[pos=\umlBottomRight{pack}, stereotype=Annotation, posDelta={-5, 5},
				refpoint=t]{FaultInjects}{}{}
			\umlClass[pos=\umlBottom{pack}, stereotype=class, posDelta={0, 5},
				refpoint=t]{Controller}{}{}		
			\umlInstance{AddRunTimeAnnotation}{FaultInj}	
			\umlInstance{AddRunTimeAnnotation}{FaultInjects}
			\umlInstance{Controller}{FileProcessor}
			\umlInstance{AddRunTimeAnnotation}{FileProcessor}									
		}% End of diagram
	%	\captionsetup{list=false}
	\caption{UML Dynamische Konfiguration}
 	\label{AddRunTimeAnnotationUML}
\end{figure}


\subsubsection{Voraussetzungen/Libraries/Konfiguration}
Für die Änderungen am Bytecode wurden die Bibliotheken Javassist und Tools verwendet. 

\begin{enumerate}
	\item Javassist: Durch Javassist (Java Programming Assistant) ist eine einfache M\"oglichkeit der Bytecode Manipulation gegeben. Diese Bibliothek erlaubt es Klassen zu ver\"andern, welche bereits von der JVM geladen wurden. Die Annotationen des \courier{StreamProcessor} konnten dadurch mit neuen Annotationen überschrieben werden. 
	\item Tools: Bereits ab Java 1.4 wird die Tools Library im Ordner lib der Java Version mitgeliefert. Sie wurde im Build Path eingebunden, um die Klasse \courier{HotSwapper} aus Javassist verwenden zu können. Dies war für den Reload der modifizierten Class-Datei in die JVM notwendig. Bei der Programmausführung müssen folgende VM-Argumente angegeben werden.
	\begin{itemize}
		\item Java 1.4: \\-Xdebug -Xrunjdwp:transport=dt\_ socket, server=y, suspend=n, address=8000
		\item ab Java 5:\\ -agentlib:jdwp=transport=dt\_ socket, server=y, suspend=n, address=8000
	\end{itemize}		
\end{enumerate}

