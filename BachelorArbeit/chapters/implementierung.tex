\label{Implementierung}
\chapter{Implementierung}\vspace{1cm}
Das folgende Kapitel zeigt die Implementierungsdetails der im Design-Kapitel erläuterten Komponenten. Es werden hierfür die wichtigsten Funktionen beschrieben und zum besseren Verständnis der Code anhand von kurzen Ausschnitten aufgezeigt.\\
Die einzelnen Klassen wurden nach ihrem Aufgabengebiet in verschiedene Java-Pakete eingeteilt. Zunächst wird das Paket \courier{faults} betrachtet. Dessen Inhalte sind alle Klassen die Fehlerwerte repr\"asentieren und daher von den folgenden Klassen verwendet werden.\\
Das Paket \courier{processor} bildet den Hauptteil des Kapites. Es beinhaltet die Klassen  \courier{Context}, \courier{StreamProcessor}, also auch die Klasse \courier{AddRuntimeAnnotation}. Alle drei Klassen sind eng aneinander gekoppelt und für die Hauptfunktionalität der Anwendung zuständig. \\
Die einzelnen Injektions-Logiken wurden in das Paket \courier{strategies} gepackt. In diesem Abschitt steht das verwendete Pattern, welches für die Implementierung der Injektion verwendet wurde, im Fokus. Die vorhandenen Strategien werden nur kurz, anhand einer ausgewählten Strategie exemplarisch vorgestellt.\\
Die Implementierungen des \courier{Controllers} und des \courier{MainServers} sind Bestandteil des Paketes \courier{jmx}. In diesem Abschnitt wird die verwendete JMX-Technologie detailliert beschrieben und anschlie\ss end dessen Verwendung anhand verschiedener JMX-Client Implementierungen aufgezeigt.


\section{Fehlerwerte}
Die Fehlerwerte der zu injizierenden Streams sind im Paket \courier{faults} erstellt worden. Dieses Paket besteht aus einer Klasse, zwei Annotationen und einer Enumeration. In der weiteren Implementierung werden diese Konstrukte an vielen Stellen verwendet und stehen insbesondere bei der dynamischen Anpassung der Fehlerwerte im Fokus.

\begin{enumerate}
	\item FaultInj : Die Annotation \courier{FaultInj} wurde eigens definiert, um die zu injizierenden Streams als fehlerbehaftet kennzeichnen zu können. Sie besteht aus den Werten id, type, rate und blocksize. In Listing \ref{lstFaultInj} ist zu erkennen, dass alle Variablen bis auf die id einen Default-Wert besitzen. Eine Fehlermarkierung kann somit wie in Listing \ref{lstannoStream} vorgenommen werden. In diesem Fall ist zwar eine Markierung vorhanden, es werden aber vorerst keine Fehlerwerte eingestreut. Aufgrund der Eindeutigkeit der ID besitzt sie keinen Default-Wert. Der Entwickler wird somit explizit durch den Compiler zur Angabe einer ID gezwungen. \\
	  
	\lstinputlisting[style=stJava,caption={Annotation FaultInj}, label=lstFaultInj]{listings/FaultInj.java}
	
		  
\lstinputlisting[style=stJava,caption={Annotation Default-Werte}, label=lstannoStream]{listings/annoStream.java}	  
	  
	
	\item FaultInjects : F\"ur Streams, die mehr als einen Fehlerwert erhalten sollen, wurde eine weitere Annotation definiert. Da mehrere Annotationen für eine Variable nicht zulässig sind, wurde dieses Verhalten über einen Umweg erzielt. Die \courier{FaultInject}-Annotation aus Listing \ref{lstFaultInjects} besitzt einen Array vom Typ der urspr\"unglichen \courier{FaultInj}-Annotation. Der Stream wird nun einfach mit der \courier{FaultInject}-Annotation markiert und die gewünschten \courier{FaultInj}-Annotationen dem Array übergeben.\\
	
	\lstinputlisting[style=stJava,caption={Annotation FaultInjects}, label=lstFaultInjects]{listings/FaultInjects.java}
		
	\item FaultType : F\"ur die angebotenen Fehlertypen wurde ein Enum mit dem Namen \courier{FaultType} erstellt (Listing \ref{lstFaultType}). Die Fehlertypen beziehen sich, wie im Design Kapitel erl\"autert, auf die Injektions-Strategie. Soll die Anwendung um eine Fehlerstrategie erweitert werden, kann diese \"uber das Enum \courier{FaultType} dem Client zug\"anglich gemacht werden. Ausschlie\ss lich hier definierte Typen werden von der Anwendung akzeptiert.\\
	
	\lstinputlisting[style=stJava,caption={Enum FaultType}, label=lstFaultType]{listings/FaultType.java}

	\item FaultValue: Die Klasse \courier{FaultValue} dient als Wrapper-Klasse\footnote{Aufname von Daten in ein Objekt} f\"ur die Parameter einer Annotation. Die Klasse erleichtert somit die Datenübergabe an andere Klassen sowie das Auslesen der jeweiligen Fehlerparameter.
\end{enumerate}






%%
\section{Klasse StreamProcessor}
Der \courier{StreamProcessor} ist das Zentrum der Anwendung. Über diese Klasse werden zu Beginn die Daten aus der Datei oder dem Stream gelesen. Die Daten werden anschlie\ss end als einzelne Bytes an die Klasse \courier{Context} übergeben. Dieses Vorgehen ist durch Listing \ref{lstFaultPro2} umgesetzt worden.
\begin{figure}[!htb]
	\lstinputlisting[style=stJava,caption={StreamProcessor Einlesen der Daten}, label=lstFaultPro2]{listings/FileProcessor2.java}
\end{figure}
F\"ur das Einlesen der Daten wurde ein \courier{InputStream} verwendet. Dieser Stream wird von \courier{FileIO}, \courier{Sockets}, etc. unterstützt. Der fehlerbehaftete Stream muss global deklariert sein, damit er annotiert/injiziert werden kann. Die beiden M\"oglichkeiten der Annotierung sind durch Listing \ref{lstFaultPro1} gegeben.\\
Im n\"achsten Schritt wird der \courier{InputStream} durch die Methode \courier{readStream} in eine Byte-Liste ausgelesen. Aufgerufen wurde \courier{readStream} durch die Methode \courier{loadStream}, welche die erstellte Ergebnisliste bzw. den Datensatz mit einer dazugehörigen ID an das \courier{Context} Objekt übergibt. Im zweiten Schritt werden die Fehlerwerte durch die Methode \courier{setFaultsByIDToContext} unter Verwendung von Java Reflection \footnote{http://docs.oracle.com/javase/tutorial/reflect/index.html - Zugriff am 21. Juli 2012
} an das \courier{Contex} Objekt übergeben. Die Methode benötigt dafür die Fehler-ID's und die ID des betroffenen Datensatzes.
\begin{figure}[!htb]
	\lstinputlisting[style=stJava,caption={StreamProcessor Fehlermarkierung}, label=lstFaultPro1]{listings/FileProcessor1.java}
\end{figure}
Das Laden der Daten wird beim Aufruf der Methode \courier{processData} aus Listing \ref{lstFaultPro3} veranlasst. Nach dem Laden folgt die Methode \courier{injectFaults} der Klasse \courier{Context}. Diese sorgt intern daf\"ur, dass die verwalteten Daten mit den beigef\"ugten Fehlerwerten injiziert werden.\\
F\"ur das Auslesen der manipulierten Daten aus dem \courier{Context} Objekt, sowie das Schreiben in eine neue Datei wurde die innere Klasse \courier{Reducer} implementiert. Der \courier{Reducer} erstellt zun\"achst den Stream für die Ausgabe und bereitet eine neue Datei vor. Danach wird die Methode \courier{reduce} aufgerufen um durch den jeweiligen Datensatz zu iterieren und die Bytes durch den gewählten Stream in die neue Datei zu schreiben.
\begin{figure}[!htb]
	\lstinputlisting[style=stJava,caption={StreamProcessor Einlesen/Injektion/Ausgabe}, label=lstFaultPro3]{listings/FileProcessor3.java}
\end{figure}
Das Auslesen der \courier{FaultValues} ist durch die Implementierung aus Listing \ref{lstFaultPro4} realisiert worden. Diese Methode wird ebenfalls im \courier{Controller/MBeanController} verwendet, um  auf Anfrage des Clients die Fehlerwerte liefern. Die Funktion arbeitet mit der bereits erw\"ahnten Java Reflection Technologie um die injizierten Felder zu ermitteln. Es werden alle globalen Variablen durchgegangen und \"uberpr\"uft, ob die \courier{FaultInjects} Annotation präsent ist. Ist dies der Fall, werden alle \courier{FaultInj} Annotationen heraus gefiltert und deren Fehlerwerte als \courier{FaultValue} verpackt zu einer Liste beigef\"ugt. F\"ur den Fall, dass statt einer \courier{FaultInjects} Annotation nur eine einzelne \courier{FaultInj} Annotation vorhanden ist, wird im \courier{else}-Zweig die Verarbeitung analog f\"ur die einzelne \courier{FaultInj} durchgef\"uhrt.
\begin{figure}[!htb]
	\lstinputlisting[style=stJava,caption={StreamProcessor Fehlerwertrückgabe}, label=lstFaultPro4]{listings/FileProcessor4.java}
\end{figure}
%%
\section{Klasse AddRuntimeAnnotation}

Für die dynamische Konfiguration der Fehlerwerte wurde die Klasse \courier{AddRuntimeAnnotation} geschrieben. Sie enthält eine statische Methode \courier{addFaultInjAnnotationToMethod} die für die Umsetztung der Modifikation zuständig ist. Bei ihrer Ausführung ver\"andert sie die Class Datei des StreamProcessors und liefert als Resultat eine Instanz des modifizierten Objektes zurück. Die Methode wurde so konzipiert, dass mehrere Annotationen mit einem Aufruf ge\"andert werden k\"onnen. F\"ur die Modifikation nutzt die Methode verschiedene Komponenten aus der Library javassist.\\
Eine in der Implementierung aus Listing \ref{lstAddRun1} benutzte Komponente ist die Klasse \courier{Javassist.CtClass}\footnote{compile-time class}, welche die Class Datei des StreamProcessors repr\"asentiert. Um eine solche \courier{CtClass} erstellen und sp\"ater modifizieren zu k\"onnen wurde vorher ein \courier{ClassPool} Objekt erstellt. Der \courier{ClassPool} liest mittels des Schl\"usselworts \courier{get} die gew\"unschte Class Datei und konstruiert aus dieser eine \courier{CtClass}. Die \courier{CtClass} kann nun auf Bytecode-Ebene angepasst werden. Um den Bytecode des StreamProcessors updaten zu können, wurde ein Objekt vom Typ \courier{HotSwapper} erstellt. Dieses ben\"otigt lediglich einen Port für das Laden der neuen Class Datei\footnote{inter-thread communication}. Die Umsetzung der Erstellung dieser Komponenten wird in Listing \ref{lstAddRun1} gezeigt.

\begin{figure}[!htb]
	\lstinputlisting[style=stJava,caption={Initialisierung HotSwapper \& CtClass}, label=lstAddRun1]{listings/AddRuntime1.java}
\end{figure}

Als Parameter bekommt die Methode \courier{addFaultInjAnnotationToMethod} die neuen Fehlerwerte anhand eines \courier{FaultValue} Array \"ubergeben. Mittels der ID des jeweiligen FaultValues kann, wie in Listing \ref{lstAddRun2} zu sehen, das dazugeh\"orige Feld in der Klasse \courier{StreamProcessor} bestimmt werden. Die ermittelten Felder bekommen f\"ur den Manipulationsvorgang eine Instanz der Klasse \courier{Javassist.CtField} zugewiesen. 

\begin{figure}[!htb]
	\lstinputlisting[style=stJava,caption={Bestimmung injizierter Felder}, label=lstAddRun2]{listings/AddRuntime2.java}
\end{figure}

Um neue Felder, Methoden, Annotationen etc. hinzuf\"ugen zu k\"onnen wurde ein \courier{ClassFile} Objekt auf den \courier{CtClass} Class-File referenziert. Zus\"atzlich wurde ein \courier{ConstPool}\footnote{constant pool table} Objekt mit Verweis auf diesen Class-File erstellt (Listing \ref{lstAddRun3}). 
Für die neuen Annotationen wurden zwei \courier{javassist.Annotation} Objekte definiert. Eines f\"ur die \courier{FaultInjects} und das andere Objekt f\"ur die \courier{FaultInj} Annotationen. Ein \courier{AnnotationsAttribute} Objekt ist nach der Erstellung der Annotationen f\"ur die \"Ubergabe an das entsprechende \courier{ctField} notwendig. 

\begin{figure}[!htb]
	\lstinputlisting[style=stJava,caption={Neue Annotationen}, label=lstAddRun3]{listings/AddRuntime3.java}
\end{figure}

F\"ur die Erstellung der neuen Annotationen wurde zun\"achst durch die vorher bestimmten \courier{CtFelder} iteriert. Es wurde die alte Annotation bestimmt und eine neue Annotation dieses Typs (\courier{FaultInjects} oder \courier{FaultInj}) angelegt (Listing \ref{lstAddRun4}). Die erstellten Annotationen wurden gleichzeitig zur Speicherung, der constant-pool-table beigef\"ugt. 

\begin{figure}[!htb]
	\lstinputlisting[style=stJava,caption={Zuweisung des Annotationstyps}, label=lstAddRun4]{listings/AddRuntime4.java}
\end{figure}

Im \courier{if}-Zweig, also falls ein Feld mit einer \courier{FaultInjects}-Annotation pr\"asent ist, wird ein \courier{AnnotationMemberValue} Array f\"ur die Unterannotationen verwendet. Es werden zun\"achst alle Unterannotationen durchlaufen und nach der passenden ID gesucht. Falls die ID gefunden wurde, können die neuen Fehlerwerte aus dem \courier{FaultValue} Array geschrieben werden. F\"ur alle anderen Unterannotationen wurden die alten Werte erneut verwendet. Nach jedem Durchlauf wird die erstellte Unterannotation dem \courier{AnnotationMemberValue} Array \"ubergeben. Dieser repr\"asentiert die \courier{FaultInjects}-Annotation und kann im Anschluss komplett an das \courier{Annotation} Array übergeben werden (Listing \ref{lstAddRun5}).  \\
Im \courier{else}-Zweig, falls also ein Feld mit einer \courier{FaultInj}-Annotation pr\"asent ist, werden lediglich die neuen Werte in die Annotation aufgenommen. Da im anfangs erstellten \courier{CtField} Array nur Felder enthalten sein k\"onnen die ge\"andert werden sollen, muss hier auch keine weitere \"Uberpr\"ufung durchgef\"uhrt werden.\\
Zum Schluss kommt das \courier{AnnotationsAttribute} Objekt zum Einsatz. Jede Annotation wird diesem Objekt \"ubergeben und die Sichtbarkeit zur Laufzeit gesetzt. Dann kann dieses dem entsprechenden \courier{CtField} \"ubergeben werden.

\begin{figure}[!htb]
	\lstinputlisting[style=stJava,caption={Setzen der Fehlerwerte}, label=lstAddRun5]{listings/AddRuntime5.java}
\end{figure}

Nachdem die Klasse mit den neuen Annotationen versehen wurde, muss der Bytecode neu geladen werden. Hierf\"ur  wird das \courier{CtClass} Objekt in eine Class Datei transformiert und anschlie\ss end der Bytecode ausgelesen. Die Klasse \courier{javassist.HotSwapper} l\"adt nun anstelle des urspr\"unglichen, den modifizierten Bytecode erneut in die JVM (Listing \ref{lstAddRun6}).

\begin{figure}[!htb]
	\lstinputlisting[style=stJava,caption={Laden des Bytecodes}, label=lstAddRun6]{listings/AddRuntime6.java}
\end{figure}

%constant pool table, an array of variable-sized constant pool entries, containing items such as literal numbers, strings, and references to classes or methods. Indexed starting at 1, containing (constant pool count - 1) number of entries in total (see note).
%
%The constant pool table is where most of the literal constant values are stored. This includes values such as numbers of all sorts, strings, identifier names, references to classes and methods, and type descriptors. All indexes, or references, to specific constants in the constant pool table are given by 16-bit (type u2) numbers, where index value 1 refers to the first constant in the table (index value 0 is invalid).







%
\section{Context}

Die Klasse \courier{Context} ist f\"ur die Datenverwaltung/-manipulation verantwortlich. In Listing \ref{lstContext1} ist die Methode \courier{write} zu sehen die vom \courier{StreamProcessor} zur \"Ubergabe der Datens\"atze aufgerufen wird. Als Prämisse dafür gilt die Serialisierbarkeit der Daten. Zur Erfassung wurde eine \courier{HashMap} gewählt, über die eine einfache Verwaltung der Datensätze anhand der ID möglich ist.

\begin{figure}[!htb]
	\lstinputlisting[style=stJava,caption={Context: Einlesen der Datensätze}, label=lstContext1]{listings/Context1.java}
\end{figure}

Um die Daten mit Fehlern zu injizieren, wurde die Methode \courier{getStrategy} aus Listing \ref{lstContext2} erstellt. Die Methode bekommt einen Datensatz mit dem dazugehörigen Fehlerwert als Parameter \"ubergeben. Anhand des Fehlertyps wird die zugrundeliegende Strategie ausgewählt. 
Neue Strategien müssen in diese Methode integriert werden.

\begin{figure}[!htb]
	\lstinputlisting[style=stJava,caption={Context: Strategieauswahl}, label=lstContext2]{listings/Context2.java}
\end{figure}

Die gew\"ahlte Strategie kann nun in der Methode \courier{injectFaults} f\"ur die Dateninjektion verwendet werden (Listing \ref{lstContext3}). Die Variable \courier{registeredFaultInjectors} ist eine Liste, bestehend aus Tupeln von IDs\footnote{Referenz auf Datensätze} und dazugeh\"origen \courier{FaultValues}. Im Gegensatz zu den Datens\"atzen der \courier{HashMap} sind bei dieser Liste die IDs nicht eindeutig, da jeder Datensatz mehrere \courier{FaultValues} haben kann. Die gew\"ahlte Strategie bekommt diese Werte \"ubergeben und ruft ihrerseits die Methode \courier{runInjection} auf. 

\begin{figure}[!htb]
	\lstinputlisting[style=stJava,caption={Context: Injektionsausführung}, label=lstContext3]{listings/Context3.java}
\end{figure}
%
\section{Fehler Injektion}

Die Implementierung der Fehlerinjektionen wurde mittels einer abgewandelten Form des Strategy Patterns realisiert. Die abstrakte Klasse \courier{InjectionStrategy} dient als Oberklasse aller  konkreten Strategien. Sie stellt für ihre Unterklassen die beiden Funktionen aus Listing \ref{lstFaultInjector1} bereit. Die Methode \courier{isInjected} ist f\"ur jede Strategie gleich und wird daher bereits mit Funktionalit\"at an die Unterklassen vererbt. \courier{isInjected} ist f\"ur den Funken Zufall in den einzelnen Injektionen verantwortlich. Anhand der Fehlerrate, die zwischen 0 und 1 liegen muss und einem Zufallswert der ebenfalls zwischen 0 und $<$1 liegt, wird entschieden ob ein Bit bzw. Block injiziert werden soll.\\
Die zweite Methode ist mit dem Schl\"usselwort \courier{abstract} deklariert. Die Implementierung der eigentlichen Algorithmen finden sich erst in der Methode \courier{runInjection} der jeweiligen Unterklassen wieder.\\
Verwendet wird dieses Muster in der bereits beschriebenen Klasse \courier{Context} durch die Methode \courier{getInjectionStrategy}. Diese Methode erzeugt eine Member-Variable aus der abstrakten Klasse \courier{InjectionStrategy}, die mit einer Referenz auf die gew\"unschte Logik belegt ist. Somit kann eine konkrete Strategie beliebig eingebunden als auch ausgetauscht werden. Die Implementierung ist dadurch sehr leicht erweiterbar. Insgesamt benötigen neue Strategien folgende Anpassungen/Erweiterungen:

\begin{itemize}
		\item einen eigenen Typ im \courier{Enum FaultType}
		\item müssen von der aktrakten Klasse \courier{InjectionStrategy} erben und dessen abstrakte Methode \courier{runInjection} implementieren.
		\item müssen in die Methode \courier{getStrategy} der Klasse \courier{Context} integriert werden
	\end{itemize}
	
\begin{figure}[!htb]
	\lstinputlisting[style=stJava,caption={InjectionStrategy}, label=lstFaultInjector1]{listings/FaultInjector1.java}
\end{figure}



\section{JMX Implementierung}

\subsection{Controller/MainServer}

Als Schnittstelle zwischen Benutzer und Anwendung wurde die Klasse \courier{Controller} eingef\"uhrt. Sie wird vom Anwender \"uber den sogenannten MBean-Server angesprochen. Die Klasse \courier{Controller} ist in der Anwendung f\"ur die Steuerung der Fehlerinjektion und weiteren Funktionalitäten verantwortlich. Der Client bekommt die f\"ur ihn vorgesehene Funktionalit\"at durch das Interface \courier{ControllerMBean} aus Listing \ref{lstController2}. Das Interface stellt alle nach au\ss en sichtbaren Methoden bereit und muss vom \courier{Controller} implementiert werden. Der \courier{Controller} ist somit \"uber das Interface ansprechbar, alle weiteren Implementierungsdetails bleiben nach au\ss en verborgen.

\begin{figure}[!htb]
	\lstinputlisting[style=stJava,caption={ControllerMBean}, label=lstController2]{listings/Controller2.java}
\end{figure}

Die Implementierung des \courier{MainServer} ist in Listing \ref{lstController3} dargestellt. Um die Java Management Extensions nutzen zu k\"onnen, muss zun\"achst ein \courier{MBeanServer} Objekt durch den Aufruf der Methode \courier{getPlatformMBeanServer} der Klasse \courier{java.lang.management.ManagementFactory} erstellt werden. Danach wird ein \courier{ObjectName} f\"ur das zu erstellende \courier{Controller} MBean definiert. Dieses besteht aus einer Domain und einer Liste von key-properties. Die Domain ist das \courier{package} in dem sich das zu erstellende MBean befindet. In Listing \ref{lstController3} ist dies das \courier{package jmx}. Die key-property spezifiziert den Typen des Objektes, in diesem Fall der Typ \courier{Controller}. Im Anschluss wird eine Objektinstanz des \courier{Controllers} erzeugt. Dieses kann nun als MBean mit dem \courier{ObjectName} Objekt am \courier{MBeanServer} registriert werden. Nach dem Starten der Klasse \courier{MainServer} wird auf die Aufrufe der im \courier{ControllerMBean} definierten Methoden gewartet, welche dann \"uber den \courier{Controller} ausgef\"uhrt werden.

\begin{figure}[!htb]
	\lstinputlisting[style=stJava,caption={MainServer}, label=lstController3]{listings/Controller3.java}
\end{figure}

 F\"ur die M\"oglichkeit einer zus\"atzlichen RMI Connection wurde die Implementierung aus Listing \ref{lstController1} verwendet. Falls keine Parameter mitgegeben wurden, wird die Klasse \courier{Mainserver} wie bisher gestartet. Andernfalls existiert die Klasse \courier{JMXServiceURL} aus \courier{javax.manage-\\ment.remote}, der sich Hostname und Port als Parameter \"ubergeben lassen. Die somit erstellte URL wird mit dem MBean Server \"uber einen \courier{JMXConnectorServer} verkn\"upft.

\begin{figure}[!htb]
	\lstinputlisting[style=stJava,caption={MainServer RMI}, label=lstController1]{listings/Controller1.java}
\end{figure}

\subsection{Client Implementierung}

JMX-Clients gibt es bereits in vielen Variationen. Sehr beliebt sind die JConsole, Java-, Shell- oder HTML-Clients\footnote{http://www.servletsuite.com/jmx/jmx-html.htm - Zugriff am 09. September 2012}. Im Folgenden werden ein Shell-Skript und eine Java Client Implementierung vorgestellt.

\subsubsection{Shell-Skript}
 F\"ur das Bash-Script wurde die jmxterm-1.0 library verwendet. Diese bietet eine intuitive Handhabung und ermöglicht einen einfachen Verbindungsaufbau. Der implementierte Client besteht aus einer bash-Datei f\"ur den Verbindungsaufbau und dem Starten der jmxterm.jar mit einer Ausf\"uhrungsliste. F\"ur eine RMI-Connection gen\"ugt der Aufruf der jmxterm library mit der entsprechenden Service URL. Die lokale Verbindung aus Listing \ref{lstscript} muss zun\"achst den erforderlichen \courier{MainServer} Process\footnote{jps: Java Virtual Machine Process Status Tool} bestimmen und diesen anschlie\ss end mit dem Schl\"usselwort open der Ausf\"uhrungsliste beif\"ugen. Der open Befehl \"offnet entweder eine neue JMX-Session oder gibt die aktuelle Verbindung wieder.
Eine Ausf\"uhrungsliste ist eine zus\"atzliche Text-Datei, die Anweisungen f\"ur den direkten Verbindungsaufbau zum \courier{MainServer} und alle gew\"unschten Methodenaufrufe erfasst.  

\begin{figure}[!htb]
	\lstinputlisting[style=stBash,nolol=true]{listings/run.sh}
	\lstinputlisting[caption={Bash-File und Ausf\"uhrungsliste}, captionpos=b, label=lstscript]{listings/script.txt}
\end{figure}

\subsubsection{Java Client}

Die Erstellung eines Java Clients verl\"auft relativ analog wie das eben beschriebene Beispiel. Zun\"achst wird wieder die JMX Verbindung per RMI oder Lokal aufgebaut. Danach wird ein \courier{MBeanServerConnection} Objekt erstellt \"uber dessen alle Methodenaufrufe ausgef\"uhrt werden. Sollen beispielsweise ein \courier{FaultValue} gesetzt werden, wird die Methode \courier{setFaultValue} in ein \courier{Attribute} Objekt verpackt und anschlie\ss end der \courier{MBeanServerConnection} mittels dessen Methode \courier{setAttribute} \"ubergeben. Getter-Methoden wie \courier{getCurrentFaults} k\"onnen analog mit \courier{getAttribute} ausgewertet werden. Handelt es sich bei einer Methode nicht um ein \courier{Attribute}\footnote{getter-/setter-Methoden mit nur einem Parameter} wird die Methode \courier{invoke} der \courier{MBeanServerConnection} verwendet. Diese bekommt als Parameter das \courier{ObjectName} Objekt, den Methodennamen und dessen Parameterwerte, als auch die Methodensignatur.

\newpage
\phantom{dd}
\begin{figure}[t!]
	\lstinputlisting[style=stJava,caption={Java Client}, captionpos=b, label=lstclient]{listings/Client1.java}
\end{figure}

%\subsubsection{HTML Adapter}
%
%Eine Connection \"uber einem HTML Adapter wird wie in Listing \ref{lsthtml} implementiert. Das Muster ist wieder dem der vorangegangen Clients sehr \"ahnliche. Zun\"achst wird eine Server-Connection aufgebaut und die Domain des Controller gew\"ahlt. Anschlie\ss end kann der HtmlAdaptorServer und das ObjectName Objekt sich am MBeanServer registrieren. Durch den Aufruf von http://localhost:9992 im Webbrowser kann \"uber den Adapter aus Listing \ref{lsthtml}, auf das Controller MBean zugegriffen werden.
%
%\begin{figure}[!htb]
%	\lstinputlisting[style=stJava,caption={Java Client}, captionpos=b, label=lsthtml]{listings/html.java}
%\end{figure}



%
%
%
%
%
%
