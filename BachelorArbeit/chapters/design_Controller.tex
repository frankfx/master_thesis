\subsection{JMX Agent}
\label{controller}

F\"ur den Zugriff auf dem \courier{Controller} und dessen Funktionalitäten wurde ein JMX Agent verwendet. Dieser bietet die M\"oglichkeit Komponenten zur Laufzeit zu managen, eine standardisierte Schnittstelle zum Application Server zu haben oder die Anwendungsfunktionalit\"at leichter administrierbar zu machen. \\
Für die Verwendung des JMX Agent musste ein management Interface erstellt werden. Das management Interface stellt dem Client ausgewählte Funktionalitäten der gemanagten Komponente zur Verfügung. In dieser Anwendung erfüllt das Interface \courier{MBeanController} aus Abbildung \ref{MBeanInterfaceUML} diesen Aufgabenteil. Die durch den JMX Agent verwaltete Klasse ist der \courier{Controller}, welcher das Interface \courier{MBeanController} implementieren muss. Insgesamt werden in dieser Anwendung sieben Methoden des \courier{Controllers} über den \courier{MBeanController} bereitgestellt. In Abbildung \ref{MBeanInterfaceUML} ist diese Klassenhierarchie mit den Funktionalitäten dargestellt. Für das Design des management Interface verlangte der JMX Agent besondere Namenskonventionen. Es wird grundsätzlich zwischen Attributen und Methoden unterschieden. Um ein Attribute zu setzen oder abzufragen, müssen Methoden mit dem Pr\"afix \textit{set-} bzw. mit dem Präfix \textit{get-} im \courier{Controller} erstellt und im Interface angegeben werden. Alle weiteren Methoden werden als normale Operationen angesehen \cite{JMXOracale}. \\

\begin{figure}[!htb] 
\centering
		\umlDiagram[box=,border,sizeX=12cm,sizeY=12cm,ref=pack]{		
			\umlClass[pos=\umlBottom{pack}, stereotype=Class, posDelta={0ex, 22.5ex},
				refpoint=t]{Controller}{%
				}{%
					\umlMethod[visibility=+]{Controller}{}
					\umlMethod[visibility=+, type=void]{setFaultsByID}{String, String, double, long}
					\umlMethod[visibility=+, type=void]{setFaultsByID}{FaultValue[]}
					\hspace{0.5cm}$\dots$
				}				
			\umlClass[pos=\umlTop{pack}, stereotype=Interface, posDelta={0ex, -2ex},
				refpoint=t]{ControllerMBean}{%
				}{%
					\umlMethod[visibility=+, type=void]{setFaultsByID}{String, String, double, long}
					\umlMethod[visibility=+, type=void]{setFaultsByID}{FaultValue[]}
					\umlMethod[visibility=+, type=void]{setUseObjectOutputStream}{boolean}
					\umlMethod[visibility=+, type={String[]}]{getPossibleIDs}{}
					\umlMethod[visibility=+, type={String[]}]{getCurrentFaults}{}
					\umlMethod[visibility=+, type=void]{runInjection}{String}
					\umlMethod[visibility=+, type=void]{runInjection}{InputStream}
				}
				\umlSubclass{Controller}{ControllerMBean}											
		}% End of diagram
	%	\captionsetup{list=false}
	\caption[UML MBean Interface]{MBean Interface und verwaltete Controller Klasse}
 	\label{MBeanInterfaceUML}
\end{figure}