\phantomsection
\addcontentsline{toc}{chapter}{ZUSAMMENFASSUNG} % Wenn sie im Inhaltsverzeichnis auftauchen soll
\chapter*{Zusammenfassung}

In vielen Konzernen nehmen die Kosten des rapide wachsenden Energieverbrauchs einen immer h\"oheren Stellenwert ein. Gerade gro\ss e Rechenzentren ben\"otigen stetig h\"ohere Rechenleistungen, um die wachsende Zahl an neuen Online-Services bereitstellen zu k\"onnen. Mit den zus\"atzlich stetig steigenden Energiekosten verst\"arkt sich immer mehr die Notwendigkeit, in diesem Bereich Einsparungsm\"oglichkeiten vorzunehmen. Die Forcierung entsprechend energieeffizienter Technologien ist der Schl\"ussel zum Erfolg. Diese Bachelorarbeit baut auf dem Paradigma \textit{Accuracy Awareness}, zur energieeffizienten Verarbeitung von ``Big Data'' in Rechenzentren, auf. \textit{Accuracy Awareness} bedient sich der Eigenschaft vieler Anwendungen,  fehlertolerant bzw. unempfindlich gegen\"uber niedrigen Fehlerraten zu sein. Handels\"ubliche Hardware verwendet jedoch einen hohen Anteil ihres Energieverbrauchs zur Garantie von Fehlerfreiheit. Durch das Anpassen der Fehlerfreiheit auf ein akkurates Ma\ss, sollen an diesem Punkt Einsparungen erzielt werden. In der erstellten Arbeit wurde zu diesem Zweck ein Tool zur Fehlerinjektion f\"ur Datenströme geschrieben. Die Anwendung erlaubt es Streams, die fehlerbelastet sein d\"urfen, um eine spezielle Fehler-Annotation zu erweitern. In die eingelesenen Daten des markierten Streams werden durch die Annotation und diverser Injektionsstrategien Fehler eingestreut und somit die Daten manipuliert. Ein wesentlicher Bestandteil der Arbeit war die dynamische Regulierbarkeit der Fehlerinjektion zur Laufzeit.

\bigskip

