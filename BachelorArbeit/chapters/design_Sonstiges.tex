\subsection{Ausnahmebehandlungen}

Um mögliche Fehleingaben abzufangen, wurden drei eigene Exceptions definiert. Die Ausnahmebehandlungen werden durch folgende Punkte ausgelöst:
	\begin{enumerate}
		\item unbekannte ID : Falls die übergebene ID keinen Stream zuzuordnen ist, kann auch keine Annotation ersetzt werden. In diesem Fall wird eine \courier{NoSuchIDException} geworfen
		\item unbekannter Typ : Falls die gewählte Fehlerstrategie nicht existiert, wird eine \courier{NoSuchTypeException} geworfen.
		\item ungültige Fehlerrate : Die Fehlerrate liegt zwischen 0 und 1. Falls der übergebene Wert au\ss erhalb dieser Grenzen liegt, kann die Berechnung nicht ausgeführt werden und eine \courier{RateOutOfBoundsException} wird geworfen.
	\end{enumerate}
Der \courier{Controller} bekommt somit die Fehlermeldung nach dem Setzen der Werte direkt von der Klasse \courier{AddRuntimeAnnotation} übergeben. Der Client wird anschlie\ss end vom \courier{Controller} informiert und kann entsprechend neue Werte angeben.

