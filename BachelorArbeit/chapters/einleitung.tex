\chapter{Einleitung} \vspace{1cm}
In vielen gro\ss en Unternehmen nimmt das Thema Energieverbrauch einen immer h\"oheren Stellenwert ein. Besonders der rapide Anstieg des Energiebedarfs von Rechenzentren und Serverr\"aumen in den letzten Jahren stellt ein erhebliches Kostenproblem dar. Nicht zu vergessen sind die negativen Einfl\"usse auf die Umwelt\footnote{bspw. CO2-Emission}, welche ein hoher Energieverbrauch zus\"atzlich mit sich bringt. Dennoch ben\"otigen gerade gro\ss e Rechenzentren stetig h\"ohere Rechenleistungen, um die wachsende Zahl an neuen Online-Services bereitzustellen. Allein Google verbraucht 0,01 Prozent des gesamten weltweiten Stromverbrauchs und knapp ein Prozent des Strombedarfs aller weltweiten Rechenzentren \cite{Kommey}. \\
Durch den Einsatz moderner Technologien und neuer Verfahren ist es heute m\"oglich den Energiebedarf eines Rechenzentrums deutlich zu reduzieren. Daraus resultieren wichtige Vorteile f\"ur Unternehmen und Gesellschaft:
\begin{itemize}\itemsep0pt
	\item Die Betriebskosten eines Rechenzentrums sinken erheblich
	\item Strom- und K\"uhlleistungsbedarf sinkt
	\begin{itemize}
		\item falls diese bereits im Grenzbereich liefen, k\"onnen somit teure Neuanschaffungen vermieden werden
	\end{itemize}	 
	\item positive Umweltauswirkungen  
\end{itemize}
Dem Thema Energieeffizienz kommt heute demnach eine hohe Bedeutung zu \cite{BITKOM}.\\
Diese Bachelorarbeit baut auf dem Paradigma \textit{Accuracy Awareness}, zur energieeffizienten Verarbeitung von ``Big Data'' in Rechenzentren, auf. Das hierf\"ur entwickelte Programm hat die Aufgabe Daten \"uber Streams einzulesen und \"uber deren Bytecode Fehlerwerte einzustreuen. F\"ur dieses Verfahren, im weiteren Verlauf der Arbeit als Fehlerinjektion bezeichnet,  sind verschiedene Fehlertypen mit unterschiedenlichen Injektionsalgorithmen definiert worden. Es wurde zus\"atzlich die Integration neuer Algorithmen bei den Designentscheidungen ber\"ucksichtigt. Die Fehlerinjektion konnte zun\"achst nur zur Compilezeit festgelegt werden. Im zweiten Schritt wurde die Anwendung mittels eines Agenten dynamisch regulierbar gemacht. \\
Diese Bachelorarbeit gliedert sich in f\"unf Kapitel. Zuerst werden alle wesentlichen Grundlagen die zum besseren Verst\"andnis der Arbeit hilfreich sind, sowie das Paradigma \textit{Accuracy Awareness}, n\"aher beleuchtet. Im Anschluss werden verwandte Arbeiten aus diesem Bereich vorgestellt, die verschiedenen Ans\"atze diskutiert und mit dem entwickelten Programm dieser Arbeit verglichen. Den Hauptteil bilden die Kapitel Design und Implementierung. Im Designkapitel werden das gew\"ahlte Programmdesign, die Architektur sowie der allgemeine Funktionsablauf der Anwendung detailliert beschrieben. Anhand verschiedener Diagramme soll die Funktionsweise verdeutlicht und die Abh\"angigkeiten der einzelnen Java-Klassen aufgezeigt werden. Auf die Imlementierung dieser Designentscheidungen wird im darauffolgenden Kapitel eingegangen. Die wichtigsten Codestellen und verwendeten Muster sind in diesem Abschnitt beschrieben. Den Abschluss bildet die Evaluation, bei der anhand einer Reihe aufgestellter Messungen, die Programmeigenschaften analysiert und die Verwendbarkeit der Anwendung diskutiert wird.
