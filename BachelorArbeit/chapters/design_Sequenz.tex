\section{Sequenzieller Ablauf} 

Den zeitlichen Ablauf der Interaktion zwischen den Hauptklassen zeigt das Sequenzdiagramm aus Abbildung \ref{Sequenzdiagramm}. F\"ur die Nutzung des Programms ist die Anmeldung am MBean-Server als erster Schritt obligatorisch. Dieser stellt die Schnittstelle zum \courier{Controller} bereit, welcher alle erforderlichen Operationen f\"ur den Client verfügbar macht. Im Sequenzdiagramm aus Abbildung \ref{Sequenzdiagramm} ist zur Vereinfachung der Grafik die Verbindung zwischen Client und \courier{Controller} direkt eingezeichnet worden. \\
Nach der erfolgreichen Anmeldung am Server steht es dem Client frei welche Operation er ausf\"uhren m\"ochte. Auf der Zeitlinie aus Abbildung \ref{Sequenzdiagramm} folgt als erster Schritt eine Fehlerwertabfrage, um anhand der ID's neue Werte setzen zu können.\\
Im Anschluss wird die Default Fehlerinjektion vom Client gestartet. Der \courier{Controller} gibt den Aufruf zunächst an den \courier{StreamProcessor} weiter. Der \courier{StreamProcessor} liest die Daten ein und übergibt sie als Byteliste an ein \courier{Context}-Objekt. Die Klasse \courier{Context} wählt anhand der Fehlerwerte, in diesem Fall die Default Fehlerwerte im Quellcode, die erforderliche Strategie aus. Die Bytes werden mit Fehlern injiziert und zurück an den \courier{StreamProcessor} geschickt. 
Der Injektionsaufruf des \courier{Controllers} respektive des Clients wird nun vom \courier{StreamProcessor} durch das Schreiben der Daten in eine vorher angelegte Ausgabedatei zum Abschluss gebracht. \\
Eine Fehlerinjektion mit Streamwechsel ist in Abbildung \ref{Sequenzdiagramm} unter ``RunInjection mit Streamwechsel'' dargestellt. Zuerst wird ein anderer Ausgabestream gewählt. Diese Operation kann zeitlich auch später folgen, muss jedoch vor dem Injektionsaufruf ausgeführt werden. Danach wurden die Fehlerwerte neu gesetzt. Das Setzen der Fehlerwerte kann zu verschiedenen Ausnahmebehandlungen führen. Im Diagramm wurde eine falsche ID übergeben. Der Controller bekommt daraufhin bei der Modifikation der Annotationen eine Fehlermeldung übergeben, welche an den Client weitergereicht wird. Im Falle einer korrekten Übergabe der Fehlerwerte erhält der Controller einen modifizierten \courier{StreamProcessor}. Anschlie\ss end kann die Injektion vom Client veranlasst werden. Der Controller ruft an dieser Stelle die entsprechende Methode des modifizierten \courier{StreamProcessor} auf. Die weitere Verarbeitung geschieht ananlog zur Default Injektion.



\begin{figure}[!htb] 
\centering

\begin{sequencediagram}
\newthread{cl}{:Client}
\newinst[2]{contr}{:Controller}
\newinst[1]{fp}{:StreamProcessor}
\newinst[1]{conte}{:Context}
\newinst{ms}{:MainServer}

\begin{call}
	{cl}	{Anmeldung}{ms}{}
\end{call}

\begin{sdblock}{getCurrentFaults}{}

\begin{call}
	{cl}{gib Fehlerwerte}{contr}{aktuelle Fehlerwerte}
\end{call}

\end{sdblock}

\begin{sdblock}{RunInjection}{}

\begin{call}
	{cl}	{Injektion}{contr}{}
	\begin{call}
		{contr}{Injektion}{fp}{schreibe Daten}
		\begin{call}
			{fp}{übergabe Bytes}{conte}{injiz. Daten}
		\end{call}
	\end{call}
\end{call}

\end{sdblock}


\begin{sdblock}{RunInjection}{mit Streamwechsel}

\begin{messcall}
	{cl}{setze Ausgabestream}{contr}
\end{messcall}

\begin{call}
	{cl}	{setze Fehlerwerte}{contr}{Error ID falsch}
\end{call}

\begin{messcall}
	{cl}	{setze Fehlerwerte}{contr}{}
	\begin{call}
		{contr}{modif. StreamPr.}{fp}{}
	\end{call}
\end{messcall}


\begin{call}
	{cl}	{Injektion}{contr}{}
	\begin{call}
		{contr}{Injektion}{fp}{schreibe Daten}
		\begin{call}
			{fp}{übergabe Bytes}{conte}{injiz. Daten}
		\end{call}
	\end{call}
\end{call}
\end{sdblock}

\end{sequencediagram}

	\caption[Sequenzdiagramm]{Sequenzdiagramm}
 	\label{Sequenzdiagramm}
\end{figure}




%\newpage
%
%\begin{sequencediagram}
%\newthread[red]{r}{:Red}
%\newthread[green]{g}{:Green}
%\newthread[blue]{b}{:Blue}
%\tikzstyle{inststyle}+=[top color=yellow,bottom
%color=gray]
%\newinst{y}{:Yellow}
%\tikzstyle{inststyle}+=[bottom color=white,top
%color=white,rounded corners=3mm]
%\newinst{o}{:Rounded}
%\end{sequencediagram}
%
%\begin{sequencediagram}
%\newthread{ss}{:Client}
%\newinst{ctr}{:Controller}
%\newinst{ps}{:PhysicsServer}
%\newinst[1]{sense}{:FileProcessor}
%\begin{call}{ss}{Initialize()}{sense}{}
%\end{call}
%\begin{sdblock}{RunLoop}{Themainloop}
%\begin{call}{ss}{StartCycle()}{ctr}{}
%\begin{call}{ctr}{ActAgent()}{sense}{}
%\end{call}
%\end{call}
%\begin{call}{ss}{Update()}{ps}{}
%\begin{messcall}{ps}{PrePhysicsUpdate()}{sense}{state}
%\end{messcall}
%\begin{sdblock}{PhysicsLoop}{}
%\begin{callself}{ps}{PhysicsUpdate()}{}
%\end{callself}
%\end{sdblock}
%\begin{call}{ps}{PostPhysicsUpdate()}{sense}{}
%\end{call}
%\end{call}
%\begin{call}{ss}{EndCycle()}{ctr}{}
%\begin{call}{ctr}{SenseAgent()}{sense}{}
%\end{call}
%\end{call}
%\end{sdblock}
%\end{sequencediagram}

