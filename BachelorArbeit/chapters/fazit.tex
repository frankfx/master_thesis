\chapter{Fazit}\vspace{1cm}
%\addcontentsline{toc}{chapter}{Faziz} % Wenn sie im Inhaltsverzeichnis auftauchen %soll


Aufgrund der st\"andig steigenden Anforderungen f\"ur Server und Rechenzentren und den damit verbundenen hohen Eniergiekosten, wird es zuk\"unftig immer wichtiger die knappe Ressource Strom effizient nutzen zu k\"onnen. Es existieren bereits verschiedene Ans\"atze und neue Verfahren die schon heute einen Schritt in die richtige Richtung machen. Das in dieser Bachelorarbeit erw\"ahnte Paradigma \textit{Accuracy-Aware} ist einer dieser Ans\"atze, auf dessen Grundlage die beschriebene Applikation beruht. Die Anwendung soll f\"ur den Bereich der Identifikation von fehlerbehafteten Daten des Paradigmas eingesetzt werden.\\
Zu diesem Zweck wurde eine Lösung entwickelt, die es erlaubt, Streams mit speziellen Annotationen zu versehen, um \"uber diese Fehlerwerte in die eingelesen Daten zu streuen. Eine wichtige Eigenschaft der Anwendung sollte die dynamische Regulierbarkeit zur Laufzeit jener Fehlerwerte sein. Die definierten Fehler reichen von diversen Fehlertypen, einer flexiblen Fehlerrate von 0\% bis 100\% bis zu der M\"oglichkeit ganze Bl\"ocke in die Injektion einzubeziehen. \\
Bei der Implementierung wurde versucht m\"oglichst Modular zu arbeiten um eine sp\"atere Erweiterbarkeit des Programms zu gew\"ahrleisten. Am deutlichsten wird dies bei der Erstellung neuer Fehlerstrategien sichtbar. Diese k\"onnen separat als eigene Klasse implementiert und mit wenig Aufwand in die Anwendung eingeflochten werden. W\"ahrend die Annotationen verwendet wurden um Streams zu markieren die fehlerbehaftet sein können, wurde die JMX Technologie eingesetzt um die Anwendung zu überwachen und dynamisch konfigurierbar zu machen. Durch JMX wird eine einfache Schnittstelle bereitgestellt, \"uber die sich die Anwendung flexibel ausf\"uhren respektive in andere Implementierungen integrieren l\"asst.\\
Die Verwendbarkeit der implementierten L\"osung wurde im Kapitel Evaluation durch verschiedene Messungen analysiert. Dabei wurde der Fokus auf Laufzeit, CPU-Auslastung und Speicherverbrauch gelegt.