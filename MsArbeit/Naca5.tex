\documentclass[12pt]{scrreprt}
\usepackage{amsmath}
\usepackage{amssymb}
\usepackage[ngerman]{babel}

\newtheorem{defi}{Definition}[chapter]
\newtheorem{satz}{Satz}[chapter]
\newtheorem{bsp}{Beispiel}[chapter]

\begin{document}
\chapter{Einleitung}

Um den maximalen Auftrieb von Tragfl\"achenprofilen zu erh\"ohen, wurde zus\"atzlich die 5er Naca Serie entwickelt. Ein NACA 5 Profil hat die Form LPQXX (beispielsweise NACA 23009) und wird wie im Folgenden definiert. Hierbei ist zu beachten, dass die ersten beiden Ziffern zur sp\"ateren Berechnung umgrechnet werden.

\begin{enumerate}
	\item Ziffer: Wert zur Berechnung des optimalen Auftriebskoeffizienten bei optimalem Anstellwinkel
		\begin{itemize}
			\item $cl = L * 0.15 $
		\end{itemize}	
	\item Ziffer: Position der gr\"o\ss ten W\"olbung entlang der Sehnenlinie, beginnend bei der leading edge
		\begin{itemize}
			\item $p = P * 5 $
		\end{itemize}
	\item Ziffer: einfache oder gespiegelte Kr\"ummung
		\begin{itemize}
			\item $0$ oder $1$
		\end{itemize}
	\item Ziffer und 5. Ziffer: maximale Dicke in \% zur Sehnenl\"ange
		\begin{itemize}
			\item $t = XX$
		\end{itemize}	
\end{enumerate}

\noindent F\"ur das obige NACA 23009 w\"urde dies folgendes bedeuten: 

\begin{bsp}
NACA 23009
\begin{itemize}
	\item[L] = 2 $\rightarrow$ 2 * 0.15 $\rightarrow$  Auftriebskoeffizient = 0.3
	\item[P] = 3 $\rightarrow$ 3 * 5.0 $\rightarrow$  Position bei = 15\%
	\item[Q] = 0 $\rightarrow$ normale W\"olbung
	\item[XX] = 09 $\rightarrow$ Dicke = 9 \%
\end{itemize}
\end{bsp}

Die Konstruktion eines NACA 5 Profils sieht zwei F\"alle vor. Die ersten beiden Gleichungen beschreiben werden verwendet, wenn das Q gleich 0 ist, also ein Profil mit normaler W\"olbung kontruiert werden soll. Die letzten bewirken im Fall, dass Q gleich 1 ist, eine gespiegelte W\"olbung.\\

W\"olbung (normal)

\begin{equation}
     y_c = \left\{ \begin{array}{ll}
			\frac{k_1}{6}(x^3 - 3mx^2 + m^2 (3-m)x), & 0 \leq x \le m \\[0.5cm]
         		\frac{k_1m^3}{6}(1-x), & m \leq x \leq 1
         	\end{array}\right.
\end{equation}

Anstieg (normal)

\begin{equation}
     \frac{dy_c}{dx} = \left\{ \begin{array}{ll}
					\frac{k_1}{6}(3x^2 - 6mx + m^2 (3-m)), & 0 \leq x \le m \\[0.5cm]
         				-\frac{k_1m^3}{6}, & m \leq x \leq 1
         			\end{array}\right.
\end{equation}

W\"olbung (gespiegelt)

\begin{equation}
     y_c = \left\{ \begin{array}{ll}
			\frac{k_1}{6}\left((x-m)^3 - \frac{k_2}{k_1} (1-m)^3x-m^3x+m^3\right), & 0 \leq x \le m \\[0.5cm]
         		\frac{k_1}{6}\left(\frac{k_2}{k_1}(x-m)^3 - \frac{k_2}{k_1} (1-m)^3 x -m^3x + m^3\right), & m \leq x \leq 1
         	\end{array}\right.
\end{equation}

Anstieg (gespiegelt)

\begin{equation}
     \frac{dy_c}{dx} = \left\{ \begin{array}{ll}
					\frac{k_1}{6}\left(3(x-m)^2 - \frac{k_2}{k_1} (1-m)^3 - m^3\right), & 0 \leq x \le m \\[0.5cm]
         				\frac{k_1}{6}\left(3 \frac{k_2}{k_1} (x-m)^2 - \frac{k_2}{k_1}(1-m)^3 -m^3\right), & m \leq x \leq 1
         			\end{array}\right.
\end{equation}




In Tabelle \ref{tab:naca5} sind die Konstanten m, k1 und k1/k2 definiert. Diese wurden an der Position der maximale W\"olbung bei einem Auftriebsbeiwert von 0.3 bestimmt. Die Ergebnisse f\"ur Anstieg und W\"olbung k\"onnen linear bez\"uglich des gew\"unschten Auftriebsbeitwertes skaliert werden.\\

\begin{tabular}{lllll}\label{tab:naca5}
Beschreibung & Position max W\"olbung (p) & m & k1 & k2/k1 \\
\hline
5\% normal &  0.05 & 0.0580 & 361.400 & \\
10\% normal & 0.10 & 0.1260 &  51.640 &\\
15\% normal & 0.15 & 0.2025 &  15.957 & \\
20\% normal & 0.20 & 0.2900 &   6.643 & \\
25\% normal & 0.25 & 0.3910 &   3.230 & \\
10\% gespiegelt & 0.10 & 0.1300 &   51.990 & 0.000764 \\
15\% gespiegelt & 0.15 & 0.2170 &   15.793 & 0.00677 \\
20\% gespiegelt & 0.20 & 0.3180 &   6.520  & 0.0303 \\
25\% gespiegelt & 0.25 & 0.4410 &   3.191 & 0.1355 \\
\hline
\end{tabular}



Das Plotten geschieht mit cosinus
\begin{equation}
     \frac{x_i}{c} = \frac{1}{2} \left[ 1 - \cos \left( \frac{i * \pi}{N-1} \right) \right]
\end{equation}



\end{document}